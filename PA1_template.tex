% Options for packages loaded elsewhere
\PassOptionsToPackage{unicode}{hyperref}
\PassOptionsToPackage{hyphens}{url}
%
\documentclass[
]{article}
\usepackage{lmodern}
\usepackage{amssymb,amsmath}
\usepackage{ifxetex,ifluatex}
\ifnum 0\ifxetex 1\fi\ifluatex 1\fi=0 % if pdftex
  \usepackage[T1]{fontenc}
  \usepackage[utf8]{inputenc}
  \usepackage{textcomp} % provide euro and other symbols
\else % if luatex or xetex
  \usepackage{unicode-math}
  \defaultfontfeatures{Scale=MatchLowercase}
  \defaultfontfeatures[\rmfamily]{Ligatures=TeX,Scale=1}
\fi
% Use upquote if available, for straight quotes in verbatim environments
\IfFileExists{upquote.sty}{\usepackage{upquote}}{}
\IfFileExists{microtype.sty}{% use microtype if available
  \usepackage[]{microtype}
  \UseMicrotypeSet[protrusion]{basicmath} % disable protrusion for tt fonts
}{}
\makeatletter
\@ifundefined{KOMAClassName}{% if non-KOMA class
  \IfFileExists{parskip.sty}{%
    \usepackage{parskip}
  }{% else
    \setlength{\parindent}{0pt}
    \setlength{\parskip}{6pt plus 2pt minus 1pt}}
}{% if KOMA class
  \KOMAoptions{parskip=half}}
\makeatother
\usepackage{xcolor}
\IfFileExists{xurl.sty}{\usepackage{xurl}}{} % add URL line breaks if available
\IfFileExists{bookmark.sty}{\usepackage{bookmark}}{\usepackage{hyperref}}
\hypersetup{
  pdftitle={Reproducible Research: Peer Assessment 1},
  hidelinks,
  pdfcreator={LaTeX via pandoc}}
\urlstyle{same} % disable monospaced font for URLs
\usepackage[margin=1in]{geometry}
\usepackage{color}
\usepackage{fancyvrb}
\newcommand{\VerbBar}{|}
\newcommand{\VERB}{\Verb[commandchars=\\\{\}]}
\DefineVerbatimEnvironment{Highlighting}{Verbatim}{commandchars=\\\{\}}
% Add ',fontsize=\small' for more characters per line
\usepackage{framed}
\definecolor{shadecolor}{RGB}{248,248,248}
\newenvironment{Shaded}{\begin{snugshade}}{\end{snugshade}}
\newcommand{\AlertTok}[1]{\textcolor[rgb]{0.94,0.16,0.16}{#1}}
\newcommand{\AnnotationTok}[1]{\textcolor[rgb]{0.56,0.35,0.01}{\textbf{\textit{#1}}}}
\newcommand{\AttributeTok}[1]{\textcolor[rgb]{0.77,0.63,0.00}{#1}}
\newcommand{\BaseNTok}[1]{\textcolor[rgb]{0.00,0.00,0.81}{#1}}
\newcommand{\BuiltInTok}[1]{#1}
\newcommand{\CharTok}[1]{\textcolor[rgb]{0.31,0.60,0.02}{#1}}
\newcommand{\CommentTok}[1]{\textcolor[rgb]{0.56,0.35,0.01}{\textit{#1}}}
\newcommand{\CommentVarTok}[1]{\textcolor[rgb]{0.56,0.35,0.01}{\textbf{\textit{#1}}}}
\newcommand{\ConstantTok}[1]{\textcolor[rgb]{0.00,0.00,0.00}{#1}}
\newcommand{\ControlFlowTok}[1]{\textcolor[rgb]{0.13,0.29,0.53}{\textbf{#1}}}
\newcommand{\DataTypeTok}[1]{\textcolor[rgb]{0.13,0.29,0.53}{#1}}
\newcommand{\DecValTok}[1]{\textcolor[rgb]{0.00,0.00,0.81}{#1}}
\newcommand{\DocumentationTok}[1]{\textcolor[rgb]{0.56,0.35,0.01}{\textbf{\textit{#1}}}}
\newcommand{\ErrorTok}[1]{\textcolor[rgb]{0.64,0.00,0.00}{\textbf{#1}}}
\newcommand{\ExtensionTok}[1]{#1}
\newcommand{\FloatTok}[1]{\textcolor[rgb]{0.00,0.00,0.81}{#1}}
\newcommand{\FunctionTok}[1]{\textcolor[rgb]{0.00,0.00,0.00}{#1}}
\newcommand{\ImportTok}[1]{#1}
\newcommand{\InformationTok}[1]{\textcolor[rgb]{0.56,0.35,0.01}{\textbf{\textit{#1}}}}
\newcommand{\KeywordTok}[1]{\textcolor[rgb]{0.13,0.29,0.53}{\textbf{#1}}}
\newcommand{\NormalTok}[1]{#1}
\newcommand{\OperatorTok}[1]{\textcolor[rgb]{0.81,0.36,0.00}{\textbf{#1}}}
\newcommand{\OtherTok}[1]{\textcolor[rgb]{0.56,0.35,0.01}{#1}}
\newcommand{\PreprocessorTok}[1]{\textcolor[rgb]{0.56,0.35,0.01}{\textit{#1}}}
\newcommand{\RegionMarkerTok}[1]{#1}
\newcommand{\SpecialCharTok}[1]{\textcolor[rgb]{0.00,0.00,0.00}{#1}}
\newcommand{\SpecialStringTok}[1]{\textcolor[rgb]{0.31,0.60,0.02}{#1}}
\newcommand{\StringTok}[1]{\textcolor[rgb]{0.31,0.60,0.02}{#1}}
\newcommand{\VariableTok}[1]{\textcolor[rgb]{0.00,0.00,0.00}{#1}}
\newcommand{\VerbatimStringTok}[1]{\textcolor[rgb]{0.31,0.60,0.02}{#1}}
\newcommand{\WarningTok}[1]{\textcolor[rgb]{0.56,0.35,0.01}{\textbf{\textit{#1}}}}
\usepackage{graphicx,grffile}
\makeatletter
\def\maxwidth{\ifdim\Gin@nat@width>\linewidth\linewidth\else\Gin@nat@width\fi}
\def\maxheight{\ifdim\Gin@nat@height>\textheight\textheight\else\Gin@nat@height\fi}
\makeatother
% Scale images if necessary, so that they will not overflow the page
% margins by default, and it is still possible to overwrite the defaults
% using explicit options in \includegraphics[width, height, ...]{}
\setkeys{Gin}{width=\maxwidth,height=\maxheight,keepaspectratio}
% Set default figure placement to htbp
\makeatletter
\def\fps@figure{htbp}
\makeatother
\setlength{\emergencystretch}{3em} % prevent overfull lines
\providecommand{\tightlist}{%
  \setlength{\itemsep}{0pt}\setlength{\parskip}{0pt}}
\setcounter{secnumdepth}{-\maxdimen} % remove section numbering

\title{Reproducible Research: Peer Assessment 1}
\author{}
\date{\vspace{-2.5em}}

\begin{document}
\maketitle

\hypertarget{loading-and-preprocessing-the-data}{%
\subsection{Loading and preprocessing the
data}\label{loading-and-preprocessing-the-data}}

I unzip the file into the working directory. I read the data in and
assign it as ``act''.

\begin{Shaded}
\begin{Highlighting}[]
 \KeywordTok{unzip}\NormalTok{(}\StringTok{"activity.zip"}\NormalTok{)}
\NormalTok{ act <-}\StringTok{ }\KeywordTok{read.csv}\NormalTok{(}\StringTok{"activity.csv"}\NormalTok{)}
\end{Highlighting}
\end{Shaded}

I summarize the data to check for NA values and clean them.

\begin{Shaded}
\begin{Highlighting}[]
\KeywordTok{summary}\NormalTok{(act)}
\end{Highlighting}
\end{Shaded}

\begin{verbatim}
##      steps            date              interval     
##  Min.   :  0.00   Length:17568       Min.   :   0.0  
##  1st Qu.:  0.00   Class :character   1st Qu.: 588.8  
##  Median :  0.00   Mode  :character   Median :1177.5  
##  Mean   : 37.38                      Mean   :1177.5  
##  3rd Qu.: 12.00                      3rd Qu.:1766.2  
##  Max.   :806.00                      Max.   :2355.0  
##  NA's   :2304
\end{verbatim}

\begin{Shaded}
\begin{Highlighting}[]
\NormalTok{act <-}\StringTok{ }\KeywordTok{na.omit}\NormalTok{(act)}
\KeywordTok{summary}\NormalTok{(act)}
\end{Highlighting}
\end{Shaded}

\begin{verbatim}
##      steps            date              interval     
##  Min.   :  0.00   Length:15264       Min.   :   0.0  
##  1st Qu.:  0.00   Class :character   1st Qu.: 588.8  
##  Median :  0.00   Mode  :character   Median :1177.5  
##  Mean   : 37.38                      Mean   :1177.5  
##  3rd Qu.: 12.00                      3rd Qu.:1766.2  
##  Max.   :806.00                      Max.   :2355.0
\end{verbatim}

\hypertarget{what-is-mean-total-number-of-steps-taken-per-day}{%
\subsection{What is mean total number of steps taken per
day?}\label{what-is-mean-total-number-of-steps-taken-per-day}}

There are 53 unique dates on which data was recorded.

\begin{Shaded}
\begin{Highlighting}[]
\KeywordTok{length}\NormalTok{(}\KeywordTok{unique}\NormalTok{(act}\OperatorTok{$}\NormalTok{date))}
\end{Highlighting}
\end{Shaded}

\begin{verbatim}
## [1] 53
\end{verbatim}

I load the package dplyr in order to process the data.Then I group the
data under dates and calculate the total number of steps taken per day,
storing the results in ``daily\_steps''.

\begin{Shaded}
\begin{Highlighting}[]
\KeywordTok{library}\NormalTok{(dplyr)}
\end{Highlighting}
\end{Shaded}

\begin{verbatim}
## Warning: package 'dplyr' was built under R version 4.0.4
\end{verbatim}

\begin{verbatim}
## 
## Attaching package: 'dplyr'
\end{verbatim}

\begin{verbatim}
## The following objects are masked from 'package:stats':
## 
##     filter, lag
\end{verbatim}

\begin{verbatim}
## The following objects are masked from 'package:base':
## 
##     intersect, setdiff, setequal, union
\end{verbatim}

\begin{Shaded}
\begin{Highlighting}[]
\NormalTok{daily_steps <-}\StringTok{ }\KeywordTok{summarise}\NormalTok{(}\KeywordTok{group_by}\NormalTok{(act, date), }\DataTypeTok{total =} \KeywordTok{sum}\NormalTok{(steps))}
\end{Highlighting}
\end{Shaded}

I create a histogram of the total number of steps taken each day.

\begin{Shaded}
\begin{Highlighting}[]
\KeywordTok{hist}\NormalTok{(daily_steps}\OperatorTok{$}\NormalTok{total, }\DataTypeTok{main=}\StringTok{"Steps per Day"}\NormalTok{, }\DataTypeTok{xlab=}\StringTok{"Number of steps"}\NormalTok{, }\DataTypeTok{col =} \StringTok{"blue"}\NormalTok{)}
\end{Highlighting}
\end{Shaded}

\includegraphics{PA1_template_files/figure-latex/histogram-1.pdf}

I calculate the mean and median of the total number of steps taken per
day.

\begin{Shaded}
\begin{Highlighting}[]
\KeywordTok{mean}\NormalTok{(daily_steps}\OperatorTok{$}\NormalTok{total, }\DataTypeTok{na.rm =} \OtherTok{TRUE}\NormalTok{)}
\end{Highlighting}
\end{Shaded}

\begin{verbatim}
## [1] 10766.19
\end{verbatim}

\begin{Shaded}
\begin{Highlighting}[]
\KeywordTok{median}\NormalTok{(daily_steps}\OperatorTok{$}\NormalTok{total, }\DataTypeTok{na.rm =} \OtherTok{TRUE}\NormalTok{)}
\end{Highlighting}
\end{Shaded}

\begin{verbatim}
## [1] 10765
\end{verbatim}

\hypertarget{what-is-the-average-daily-activity-pattern}{%
\subsection{What is the average daily activity
pattern?}\label{what-is-the-average-daily-activity-pattern}}

I process the data and take the average of steps taken in each 5-minute
interval across all days.

\begin{Shaded}
\begin{Highlighting}[]
\NormalTok{by_intervals <-}\StringTok{ }\KeywordTok{aggregate}\NormalTok{(steps }\OperatorTok{~}\StringTok{ }\NormalTok{interval, act, mean)}
\end{Highlighting}
\end{Shaded}

I construct a time series plot of the 5-minute interval (x-axis) and the
average number of steps taken, averaged across all days (y-axis).

\begin{Shaded}
\begin{Highlighting}[]
\KeywordTok{plot}\NormalTok{(by_intervals}\OperatorTok{$}\NormalTok{interval, by_intervals}\OperatorTok{$}\NormalTok{steps, }\DataTypeTok{type=}\StringTok{'l'}\NormalTok{, }\DataTypeTok{main=}\StringTok{"Daily Average of Activity by 5-Min Intervals"}\NormalTok{, }\DataTypeTok{xlab=}\StringTok{"5-Min Interval"}\NormalTok{, }\DataTypeTok{ylab=}\StringTok{"Average number of steps"}\NormalTok{, }\DataTypeTok{col=}\StringTok{"red"}\NormalTok{)}
\end{Highlighting}
\end{Shaded}

\includegraphics{PA1_template_files/figure-latex/timeseries-1.pdf}

I find out the 5 minute interval with the highest daily average.

\begin{Shaded}
\begin{Highlighting}[]
\KeywordTok{which.max}\NormalTok{(by_intervals}\OperatorTok{$}\NormalTok{steps)}
\end{Highlighting}
\end{Shaded}

\begin{verbatim}
## [1] 104
\end{verbatim}

\begin{Shaded}
\begin{Highlighting}[]
\NormalTok{by_intervals[}\DecValTok{104}\NormalTok{, ]}
\end{Highlighting}
\end{Shaded}

\begin{verbatim}
##     interval    steps
## 104      835 206.1698
\end{verbatim}

\hypertarget{imputing-missing-values}{%
\subsection{Imputing missing values}\label{imputing-missing-values}}

I reread the original data and store it in actNA. I calculate the number
of missing values.

\begin{Shaded}
\begin{Highlighting}[]
\NormalTok{actNA <-}\StringTok{ }\KeywordTok{read.csv}\NormalTok{(}\StringTok{"activity.csv"}\NormalTok{)}
\KeywordTok{sum}\NormalTok{(}\KeywordTok{is.na}\NormalTok{(actNA))}
\end{Highlighting}
\end{Shaded}

\begin{verbatim}
## [1] 2304
\end{verbatim}

\hypertarget{are-there-differences-in-activity-patterns-between-weekdays-and-weekends}{%
\subsection{Are there differences in activity patterns between weekdays
and
weekends?}\label{are-there-differences-in-activity-patterns-between-weekdays-and-weekends}}

\end{document}
